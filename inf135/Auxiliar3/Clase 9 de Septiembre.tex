\documentclass{article}
\newcommand\tab[1][1cm]{\hspace*{#1}}
\usepackage[fleqn]{amsmath}
%preamble

\author{Rodrigo Rivera}
\title{Clase 9 de Septiembre}

%content

\begin{document}
\maketitle

\section{La clase}

En la clase, se completo la soluci�n del ejercicio planteado por el docente en la clase jueves, 7  de Septiembre; adem\'as se realizo una peque\~na introducci\'on al condicional if, para ello se implemento la clase Auto, el cual inicialmente pose�a como \'unico atributo aceleraci\'on y la funci\'on que nos dec\'ia cual era su posici\'on dado un tiempo, llamado "posicionActual" asumiendo que el auto part�a de la posici\'on 0 y desde el reposo.

\section{If}

El if es una estructura de control que permite tras evaluar una condici\'on ejecutar una porci\'on de c\'odigo o no. Su sintaxis es la siguiente:

if ( condici\'on) \{

	\tab c\'odigo que se ejecuta en caso de ser la condici\'on verdadera

\} else \{

	\tab c\'odigo que se ejecuta en caso de ser la condici\'on falsa

\}

La condici\'on es una expresi�n booleana, es decir que la expresi\'on solo devolver\'a dos posibles resultados, verdadero o falso. Tambi\'en se puede escribir el if sin el else

\section{Ejemplo}

Para poder mostrar el if en un ejemplo se hizo una variante al problema del auto, se le agrego el atributo velocidad m\'axima, lo cual implica que el auto se desplaza con movimiento uniformemente acelerado hasta el instante en el que llega a su velocidad m\'axima, a partir de ese momento su velocidad no variar\'ia es por ello que se desplazar\'a con movimiento uniforme. Siendo nuestra condici�n a evaluar, para la funci�n "posicionActual", si el tiempo que se recibe como par\'ametro es menor al tiempo que tarda el auto en alcanzar su velocidad m\'axima, de ser verdad, se aplica la ecuaci\'on de la posici\'on en funci\'on del tiempo para los movimientos uniformemente acelerados, en caso contrario se debe evaluar como un movimiento uniformemente acelerado hasta el tiempo de velocidad m\'axima y posteriormente como un movimiento con velocidad constante, para esta segunda formula la posici�n inicial seria el valor obtenido de aplicar la primera ecuaci\'on, la velocidad seria la velocidad m\'axima y el tiempo seria el tiempo recibido como par\'ametro menos el tiempo que tarda en llegar a su velocidad m\'axima.

\end{document}